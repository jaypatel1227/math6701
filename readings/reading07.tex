\documentclass{homework}
\usepackage{amsmath}
\DeclareMathOperator{\Mat}{Mat}
\DeclareMathOperator{\End}{End}
\DeclareMathOperator{\Hom}{Hom}
\DeclareMathOperator{\id}{id}
\DeclareMathOperator{\image}{im}
\DeclareMathOperator{\rank}{rank}
\DeclareMathOperator{\nullity}{nullity}
\DeclareMathOperator{\trace}{tr}
\DeclareMathOperator{\Spec}{Spec}
\DeclareMathOperator{\Sym}{Sym}
\DeclareMathOperator{\pf}{pf}
\DeclareMathOperator{\Ortho}{O}
\DeclareMathOperator{\diam}{diam}
\DeclareMathOperator{\Gr}{Gr}
\DeclareMathOperator{\SO}{SO}
\DeclareMathOperator{\Real}{Re}
\DeclareMathOperator{\Imag}{Im}
\DeclareMathOperator{\dR}{dR}

\newcommand{\Proj}{\mathbb{P}}
\newcommand{\RP}{\mathbb{R}P}
\newcommand{\CP}{\mathbb{C}P}

\DeclareMathOperator{\Arg}{Arg}

\newcommand{\C}{\mathbb{C}}

\DeclareMathOperator{\sla}{\mathfrak{sl}}
\newcommand{\norm}[1]{\left\lVert#1\right\rVert}
\newcommand{\transpose}{\intercal}

\newcommand{\conj}[1]{\bar{#1}}
\newcommand{\abs}[1]{\left|#1\right|}
\author{Jim Fowler}
\course{Math 6701}
\date{Week 7: Homology and cohomology}

\begin{document}
\maketitle

We've been thinking of forms mainly as ``covector fields'' but this
week, we interpret forms as integrands.  Note that forms transform
appropriately, i.e., if we walk over a parameterized submanifold twice
as fast and feed the resulting velocity vectors into a form and sum,
we end up summing quantities which are twice as large but for half as
much time.  By defining the integral of differential forms and meeting
Stokes' theorem, we will see the fundamental theorem of calculus in
its best form.

To get started, from Morita's \textit{Geometry of Differential Forms}, read
\begin{itemize}
\item 3.1 Homology of manifolds
\item 3.2 Integral of differential forms and the Stokes theorem
\item 3.3 The de Rham theorem
\end{itemize} Having seen the fundamental theorem of calculus in its
best form, it is appropriate then to consider how the fundamental
theorem might \textit{fail} on manifolds.  This failure is measured by
\textbf{de Rham cohomology}, and this week we meet \textbf{homology}
and the \textbf{de Rham theorem} which relates our computations
involving forms, the closed forms, the exact forms, back to a more
basic notion of (co)homology.  The proof of the de Rham theorem and
the Poincar\'e lemma will be postponed to next week, but even
formulating the relationship between these different formulations of
(co)homology will be interesting.  What exactly does it mean for these
two homology theories to be naturally isomorphic?

Ultimately, our study of differential forms is really getting at the
topology of smooth manifolds.  Questions about whether a form like
$d\theta$ on the circle is (despite the notation!) actually the
exterior derivative of a function is a question which detects the
``hole'' in the circle.  Cup products, which might seem hard to
understand from the perspective of singular cohomology, are as easy as
the exterior product, i.e., wedging forms.  This is a big payoff of
having taken a course in smooth manifolds: you are now armed with
something concrete enough to guide your way through a deeper study of
cohomology.

If you are looking for additional resources, I thoroughly recommend
Bott and Tu's book \textit{Differential Forms in Algebraic Topology}
which does this deeper digging into the algebraic topology.

\end{document}
