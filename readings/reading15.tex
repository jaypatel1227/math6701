\documentclass{homework}
\usepackage{amsmath}
\DeclareMathOperator{\Mat}{Mat}
\DeclareMathOperator{\End}{End}
\DeclareMathOperator{\Hom}{Hom}
\DeclareMathOperator{\id}{id}
\DeclareMathOperator{\image}{im}
\DeclareMathOperator{\rank}{rank}
\DeclareMathOperator{\nullity}{nullity}
\DeclareMathOperator{\trace}{tr}
\DeclareMathOperator{\Spec}{Spec}
\DeclareMathOperator{\Sym}{Sym}
\DeclareMathOperator{\pf}{pf}
\DeclareMathOperator{\Ortho}{O}
\DeclareMathOperator{\diam}{diam}
\DeclareMathOperator{\Gr}{Gr}
\DeclareMathOperator{\SO}{SO}
\DeclareMathOperator{\Real}{Re}
\DeclareMathOperator{\Imag}{Im}
\DeclareMathOperator{\dR}{dR}

\newcommand{\Proj}{\mathbb{P}}
\newcommand{\RP}{\mathbb{R}P}
\newcommand{\CP}{\mathbb{C}P}

\DeclareMathOperator{\Arg}{Arg}

\newcommand{\C}{\mathbb{C}}

\DeclareMathOperator{\sla}{\mathfrak{sl}}
\newcommand{\norm}[1]{\left\lVert#1\right\rVert}
\newcommand{\transpose}{\intercal}

\newcommand{\conj}[1]{\bar{#1}}
\newcommand{\abs}[1]{\left|#1\right|}
\author{Jim Fowler}
\course{Math 6701}
\date{Week 15: Fiber Bundles}

\DeclareMathOperator{\GL}{GL}
\DeclareMathOperator{\SO}{SO}

\begin{document}
\maketitle

From Morita's \textit{Geometry of Differential Forms}, read
\begin{itemize}
\item 6.1 Fiber bundle and principal bundle
\item 6.2 $S^1$ bundles and Euler classes
\item 6.3 Connections
\end{itemize} This material introduces \textbf{principal bundles},
which are not \textit{principle} but rather \textit{principal} in the
sense of ``first in order of importance.''  What exactly is principal
about a principal bundle?

A principal $\GL(V)$-bundle gives rise to an associated vector bundle,
so in that sense, these $\GL(V)$-bundles are the ``main'' object, with
the vector bundles somehow being produced from the principal bundles.
And these principal bundles have some significant advantages: we were
angry that the connection 1-form did not transform tensorially, but
for principal bundles, the connection 1-forms are honest, globally
defined 1-forms.

There are other advantages: not every fiber bundle has vector space
fibers, so armed with the notion of principal $G$-bundles, we can not
only recover the story for vector bundles (through these
``associated'' bundles) but also discuss examples beyond vector
bundles, like circle bundles.

Finally, thinking about principal $G$-bundles will enable us to
consider ``reduction of structure'' and the various relationships
between $G$-bundles and $H$-bundles.  We've seen examples of this
(like the relationship between bundles with transtiion maps in
$\GL(V)$ versus those with transition maps in $\SO(V)$) so this
framework will help us understand the situation more generally.

\end{document}
