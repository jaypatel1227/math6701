\documentclass{homework}
\usepackage{amsmath}
\DeclareMathOperator{\Mat}{Mat}
\DeclareMathOperator{\End}{End}
\DeclareMathOperator{\Hom}{Hom}
\DeclareMathOperator{\id}{id}
\DeclareMathOperator{\image}{im}
\DeclareMathOperator{\rank}{rank}
\DeclareMathOperator{\nullity}{nullity}
\DeclareMathOperator{\trace}{tr}
\DeclareMathOperator{\Spec}{Spec}
\DeclareMathOperator{\Sym}{Sym}
\DeclareMathOperator{\pf}{pf}
\DeclareMathOperator{\Ortho}{O}
\DeclareMathOperator{\diam}{diam}
\DeclareMathOperator{\Gr}{Gr}
\DeclareMathOperator{\SO}{SO}
\DeclareMathOperator{\Real}{Re}
\DeclareMathOperator{\Imag}{Im}
\DeclareMathOperator{\dR}{dR}

\newcommand{\Proj}{\mathbb{P}}
\newcommand{\RP}{\mathbb{R}P}
\newcommand{\CP}{\mathbb{C}P}

\DeclareMathOperator{\Arg}{Arg}

\newcommand{\C}{\mathbb{C}}

\DeclareMathOperator{\sla}{\mathfrak{sl}}
\newcommand{\norm}[1]{\left\lVert#1\right\rVert}
\newcommand{\transpose}{\intercal}

\newcommand{\conj}[1]{\bar{#1}}
\newcommand{\abs}[1]{\left|#1\right|}
\author{Jim Fowler}
\course{Math 6701}
\date{Week 9: Applications of cohomology}
\DeclareMathOperator{\dR}{dR}

\begin{document}
\maketitle

I've had mixed feelings about the last couple weeks -- there's a lot
happening, a lot of machinery.  And we've already seen a number of
significant applications, like the Brouwer fixed point theorem (albeit
for smooth maps!) and the fact that the two-sphere does not support a
nowhere vanishing vector field -- though we managed to get a vector
field with a \textit{single} zero on a previous homework, by paying
the price of having more complicated behavior around that zero.

This week is called ``applications\ldots'' so are there more
applications?  Of course!  Cohomology is extraordinarily important and
provides an organizing principle for a ton of mathematics.  For this
week, open Morita's \textit{Geometry of Differential Forms} and read
\begin{itemize}
\item 3.5 Applications of the de Rham theorem
\end{itemize} We've already seen \textbf{mapping degree} previously,
but this week we'll see how to use cohomology to uncover an invariant
of maps $S^3 \to S^2$, namely the \textbf{Hopf invariant}.  Having
previously seen the Hopf map as a particularly nice map from $S^3$ to
$S^2$, a natural question is: what's the Hopf invariant of the Hopf
map?  We will also meet \textbf{Massey products} which is an
opportunity to remember our prior experience with the Heisenberg
group.

There's another strand of discussion this week around the K\"unneth
formula.  This appears in a homework exercise in the \textit{next}
chapter of Morita, where the machinery of harmonic forms makes it
possible to show that the map $H_{\dR}^\star(M) \otimes
H_{\dR}^\star(N) \to H_{\dR}^\star(M \times N)$.  Given this, perhaps
the K\"unneth formula will serve as a useful bridge to what comes
next.

\end{document}
