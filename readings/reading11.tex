\documentclass{homework}
\usepackage{amsmath}
\DeclareMathOperator{\Mat}{Mat}
\DeclareMathOperator{\End}{End}
\DeclareMathOperator{\Hom}{Hom}
\DeclareMathOperator{\id}{id}
\DeclareMathOperator{\image}{im}
\DeclareMathOperator{\rank}{rank}
\DeclareMathOperator{\nullity}{nullity}
\DeclareMathOperator{\trace}{tr}
\DeclareMathOperator{\Spec}{Spec}
\DeclareMathOperator{\Sym}{Sym}
\DeclareMathOperator{\pf}{pf}
\DeclareMathOperator{\Ortho}{O}
\DeclareMathOperator{\diam}{diam}
\DeclareMathOperator{\Gr}{Gr}
\DeclareMathOperator{\SO}{SO}
\DeclareMathOperator{\Real}{Re}
\DeclareMathOperator{\Imag}{Im}
\DeclareMathOperator{\dR}{dR}

\newcommand{\Proj}{\mathbb{P}}
\newcommand{\RP}{\mathbb{R}P}
\newcommand{\CP}{\mathbb{C}P}

\DeclareMathOperator{\Arg}{Arg}

\newcommand{\C}{\mathbb{C}}

\DeclareMathOperator{\sla}{\mathfrak{sl}}
\newcommand{\norm}[1]{\left\lVert#1\right\rVert}
\newcommand{\transpose}{\intercal}

\newcommand{\conj}[1]{\bar{#1}}
\newcommand{\abs}[1]{\left|#1\right|}
\author{Jim Fowler}
\course{Math 6701}
\date{Week 11: Vector Bundles}

\begin{document}
\maketitle

From Morita's \textit{Geometry of Differential Forms}, read
\begin{itemize}
\item 4.4 Applications of the Hodge theorem
\end{itemize}
to wrap up any loose ends from last week\ldots But then, dive into
\begin{itemize}
\item 5.1 Vector bundles
\item 5.2 Geodesics and parallel translation of vectors
\end{itemize} where we meet \textbf{vector bundles}, at long last
providing precision to the vagueness of writing $TM$ for the
\textit{tangent bundle} of a smooth manifold $M$; this geometric
object gathers together all the separate $T_p M$'s, the tangent spaces
at each point $p \in M$, into a single object.  Armed with objects, we
can then talk about the morphisms---bundle maps---and even morphisms
which change the base space.  And knowing what the morphisms are, we
also learn about isomorphisms of bundles, and consequently it makes
sense to ask ``Are the tangent bundles to spheres ever trivial?''
meaning ``Is $TS^n$ ever isomorphic to the trivial $n$-dimensional
bundle over $S^n$?''

In other developments of the material, vector bundles appear very
early in the course, and there are some good reasons for this: when we
were first learning about atlases and charts, we considered
``transition functions'' moving us from one chart to another.  For a
vector bundle, we again see transition maps.

But there are also reasons to meet vector bundles \textit{after}
cohomology.  With transition maps, we'll see the ``cocycle
condition.''  And having seen de Rham cohomology with compact
supports, for vector bundles there is a new ``cohomology with compact
support in the vertical direction'' to study.

\end{document}
