\documentclass{homework}
\usepackage{amsmath}
\DeclareMathOperator{\Mat}{Mat}
\DeclareMathOperator{\End}{End}
\DeclareMathOperator{\Hom}{Hom}
\DeclareMathOperator{\id}{id}
\DeclareMathOperator{\image}{im}
\DeclareMathOperator{\rank}{rank}
\DeclareMathOperator{\nullity}{nullity}
\DeclareMathOperator{\trace}{tr}
\DeclareMathOperator{\Spec}{Spec}
\DeclareMathOperator{\Sym}{Sym}
\DeclareMathOperator{\pf}{pf}
\DeclareMathOperator{\Ortho}{O}
\DeclareMathOperator{\diam}{diam}
\DeclareMathOperator{\Gr}{Gr}
\DeclareMathOperator{\SO}{SO}
\DeclareMathOperator{\Real}{Re}
\DeclareMathOperator{\Imag}{Im}
\DeclareMathOperator{\dR}{dR}

\newcommand{\Proj}{\mathbb{P}}
\newcommand{\RP}{\mathbb{R}P}
\newcommand{\CP}{\mathbb{C}P}

\DeclareMathOperator{\Arg}{Arg}

\newcommand{\C}{\mathbb{C}}

\DeclareMathOperator{\sla}{\mathfrak{sl}}
\newcommand{\norm}[1]{\left\lVert#1\right\rVert}
\newcommand{\transpose}{\intercal}

\newcommand{\conj}[1]{\bar{#1}}
\newcommand{\abs}[1]{\left|#1\right|}
\author{Jim Fowler}
\course{Math 6701}
\date{Week 14: Applications}

\begin{document}
\maketitle

This is a shortened week---indeed, it is just one day---because of
Thanksgiving break.  Read
\begin{itemize}
\item 5.7 Applications of characteristic classes
\end{itemize}
from Morita's \textit{Geometry of Differential Forms}.

With all the effort put into building these characteristic classes,
there's a natural question: what's it all good for?  Superficially,
\textit{any} nontrivial characteristic class affirms the claim that
the bundle is not the trivial bundle\ldots but beyond that, exactly
what sort of twisting are these classes measuring?

One answer arrives with a deeper study of the Euler class which, when
integrated over the manifold, yields the Euler characteristic.  This
suggests a general mechanism for converting (top dimensional!)
cohomology classes to numbers---just integrate!  For Pontrjagin
classes, we get the Pontrjagin numbers, which ultimately connects to
cobordism, answering the question of when an $n$-manifold bounds an
$(n+1)$-manifold.


\end{document}
