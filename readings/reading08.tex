\documentclass{homework}
\usepackage{amsmath}
\DeclareMathOperator{\Mat}{Mat}
\DeclareMathOperator{\End}{End}
\DeclareMathOperator{\Hom}{Hom}
\DeclareMathOperator{\id}{id}
\DeclareMathOperator{\image}{im}
\DeclareMathOperator{\rank}{rank}
\DeclareMathOperator{\nullity}{nullity}
\DeclareMathOperator{\trace}{tr}
\DeclareMathOperator{\Spec}{Spec}
\DeclareMathOperator{\Sym}{Sym}
\DeclareMathOperator{\pf}{pf}
\DeclareMathOperator{\Ortho}{O}
\DeclareMathOperator{\diam}{diam}
\DeclareMathOperator{\Gr}{Gr}
\DeclareMathOperator{\SO}{SO}
\DeclareMathOperator{\Real}{Re}
\DeclareMathOperator{\Imag}{Im}
\DeclareMathOperator{\dR}{dR}

\newcommand{\Proj}{\mathbb{P}}
\newcommand{\RP}{\mathbb{R}P}
\newcommand{\CP}{\mathbb{C}P}

\DeclareMathOperator{\Arg}{Arg}

\newcommand{\C}{\mathbb{C}}

\DeclareMathOperator{\sla}{\mathfrak{sl}}
\newcommand{\norm}[1]{\left\lVert#1\right\rVert}
\newcommand{\transpose}{\intercal}

\newcommand{\conj}[1]{\bar{#1}}
\newcommand{\abs}[1]{\left|#1\right|}
\author{Jim Fowler}
\course{Math 6701}
\date{Week 8: Poincar\'e lemma}

\DeclareMathOperator{\dR}{dR}

\begin{document}
\maketitle

Last week was intense: we're moving at a very high speed!  This week
and the next we will continue to discuss de Rham cohomology; we'll
cycle through this material again in more detail, finally giving
proofs for some of the machinery that we previously only learned how
to use.  Why is de Rham cohomology related to singular cohomology?
Why do homotopic maps induce the same map on cohomology?  How do we
actually \textit{compute} the cohomology of any spaces we care about?

From Morita's \textit{Geometry of Differential Forms}, read
\begin{itemize}
\item 3.4 Proof of the de Rham theorem
\end{itemize}

The first goal this week is the \textbf{Pincar\'e lemma} which states
that $H_{\dR}^k(B^n) = 0$ when $k > 0$.  In words, when $k > 0$, every
closed $k$-form on the $n$-ball $B^n$ is exact.  Stated differently,
the de Rham cohomology of the ball is trivial; it's the same as
$H^k(\mbox{point})$, which is the same as $H^k(B^n)$.  One perspective
on the Poincar\'e lemma is that closed forms are locally exact; this
means that de Rham cohomology is truly detecting ``global'' features
of the manifold, precisely because it can't detect local phenomena.

But the Poincar\'e lemma is important for a deeper reason.  Knowing
that de Rham cohomology and singular cohomology agree on basic sets,
and knowing that they are both computed the same way (e.g., through
Mayer-Vietoris), ultimately means that singular cohomology and de Rham
cohomology agree.  We've already seen the map that realizes this
identification: we get a cochain from a form $\omega$ by integrating
the form on a chain.

As always, I appreciate how engaged you are during lecture.

\end{document}
