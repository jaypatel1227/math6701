\documentclass{homework}
\usepackage{amsmath}
\DeclareMathOperator{\Mat}{Mat}
\DeclareMathOperator{\End}{End}
\DeclareMathOperator{\Hom}{Hom}
\DeclareMathOperator{\id}{id}
\DeclareMathOperator{\image}{im}
\DeclareMathOperator{\rank}{rank}
\DeclareMathOperator{\nullity}{nullity}
\DeclareMathOperator{\trace}{tr}
\DeclareMathOperator{\Spec}{Spec}
\DeclareMathOperator{\Sym}{Sym}
\DeclareMathOperator{\pf}{pf}
\DeclareMathOperator{\Ortho}{O}
\DeclareMathOperator{\diam}{diam}
\DeclareMathOperator{\Gr}{Gr}
\DeclareMathOperator{\SO}{SO}
\DeclareMathOperator{\Real}{Re}
\DeclareMathOperator{\Imag}{Im}
\DeclareMathOperator{\dR}{dR}

\newcommand{\Proj}{\mathbb{P}}
\newcommand{\RP}{\mathbb{R}P}
\newcommand{\CP}{\mathbb{C}P}

\DeclareMathOperator{\Arg}{Arg}

\newcommand{\C}{\mathbb{C}}

\DeclareMathOperator{\sla}{\mathfrak{sl}}
\newcommand{\norm}[1]{\left\lVert#1\right\rVert}
\newcommand{\transpose}{\intercal}

\newcommand{\conj}[1]{\bar{#1}}
\newcommand{\abs}[1]{\left|#1\right|}
\author{Jim Fowler}
\course{Math 6701}
\date{Week 12: Geometry}

\begin{document}
\maketitle

Last week, we learned about vector bundles, which lets us do linear
algebra over a manifold.  This week is about ``geometry'' and by
geometry, I mean calculus---being able to take derivatives of sections
of a vector bundle.  We do this with a gadget called a
\textbf{connection}.  The name \textit{connection} is suggestive: the
fibers above distant points are ``connected'' via parallel transport.

It is not immediately clear that being able to take derivatives of
sections of vector bundles really amounts to doing geometry; once we
have connections, we can discuss \textbf{curvature} which is
undoubtably something geometric.  Perhaps our first experience with
curvature has to do with \textbf{holonomy} as we perform parallel
transport around closed curves.  To learn about this, read Morita's
\textit{Geometry of Differential Forms} focusing on
\begin{itemize}
\item 5.3 Connections in vector bundles
\end{itemize}

It's always important to make connections to our prior learning.  A
couple weeks ago, we first met \textit{forms}, and this week, we'll
find, after choosing a connection, we can produce a certain closed
differential form --- that is, a cohomology class -- and that class is
independent of the choice of connection.  This is our first
\textbf{characteristic class}.  Read
\begin{itemize}
\item 5.4 Pontrjagin classes
\end{itemize} to learn about these.  Characteristic classes will be a
big theme next week!

\end{document}
