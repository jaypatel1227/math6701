\documentclass{homework}
\usepackage{amsmath}
\DeclareMathOperator{\Mat}{Mat}
\DeclareMathOperator{\End}{End}
\DeclareMathOperator{\Hom}{Hom}
\DeclareMathOperator{\id}{id}
\DeclareMathOperator{\image}{im}
\DeclareMathOperator{\rank}{rank}
\DeclareMathOperator{\nullity}{nullity}
\DeclareMathOperator{\trace}{tr}
\DeclareMathOperator{\Spec}{Spec}
\DeclareMathOperator{\Sym}{Sym}
\DeclareMathOperator{\pf}{pf}
\DeclareMathOperator{\Ortho}{O}
\DeclareMathOperator{\diam}{diam}
\DeclareMathOperator{\Gr}{Gr}
\DeclareMathOperator{\SO}{SO}
\DeclareMathOperator{\Real}{Re}
\DeclareMathOperator{\Imag}{Im}
\DeclareMathOperator{\dR}{dR}

\newcommand{\Proj}{\mathbb{P}}
\newcommand{\RP}{\mathbb{R}P}
\newcommand{\CP}{\mathbb{C}P}

\DeclareMathOperator{\Arg}{Arg}

\newcommand{\C}{\mathbb{C}}

\DeclareMathOperator{\sla}{\mathfrak{sl}}
\newcommand{\norm}[1]{\left\lVert#1\right\rVert}
\newcommand{\transpose}{\intercal}

\newcommand{\conj}[1]{\bar{#1}}
\newcommand{\abs}[1]{\left|#1\right|}
\author{Jim Fowler}
\course{Math 6701}
\date{Week 13: Characteristic classes}

\begin{document}
\maketitle

As promised, this week is all about \textbf{characteristic classes}.
Because characteristic classes appear throughout mathematics (okay, at
least in algebraic geometry, in the topology of manifolds), there are
various perspectives on these objects.  In terms of how they're
constructed, we see that by choosing a connection, we can build
various closed forms related to the curvature of the connection, and
that in certain cases, the resulting cohomology classes do
\textit{not} depend on the choice of the connection.

From Morita's \textit{Geometry of Differential Forms}, read
\begin{itemize}
\item 5.5 Chern classes
\item 5.6 Euler classes
\end{itemize}

Another perspective has to do with how we build vector bundles.  A
vector bundle can be built by pulling back a certain \textbf{universal
bundle} which is as twisted as a bundle can be.  The base space of the
universal bundle---the \textbf{classifying space}---supports some
cohomology classes, and pulling back those classes provides cohomology
classes that measure the twistedness of our given bundle.

And that's a reason these characteristic classes are interesting: for
a trivial bundle, the characteristic classes vanish, so these are
cohomological invariants that measure how twisted the bundle is.
These are ``natural'' invariants, in the sense that the pull back
appropriately, i.e., the characteristic classes of the pulled back
bundle are the pulled back characteristic classes.

\end{document}
