\documentclass{homework}
\usepackage{amsmath}
\DeclareMathOperator{\Mat}{Mat}
\DeclareMathOperator{\End}{End}
\DeclareMathOperator{\Hom}{Hom}
\DeclareMathOperator{\id}{id}
\DeclareMathOperator{\image}{im}
\DeclareMathOperator{\rank}{rank}
\DeclareMathOperator{\nullity}{nullity}
\DeclareMathOperator{\trace}{tr}
\DeclareMathOperator{\Spec}{Spec}
\DeclareMathOperator{\Sym}{Sym}
\DeclareMathOperator{\pf}{pf}
\DeclareMathOperator{\Ortho}{O}
\DeclareMathOperator{\diam}{diam}
\DeclareMathOperator{\Gr}{Gr}
\DeclareMathOperator{\SO}{SO}
\DeclareMathOperator{\Real}{Re}
\DeclareMathOperator{\Imag}{Im}
\DeclareMathOperator{\dR}{dR}

\newcommand{\Proj}{\mathbb{P}}
\newcommand{\RP}{\mathbb{R}P}
\newcommand{\CP}{\mathbb{C}P}

\DeclareMathOperator{\Arg}{Arg}

\newcommand{\C}{\mathbb{C}}

\DeclareMathOperator{\sla}{\mathfrak{sl}}
\newcommand{\norm}[1]{\left\lVert#1\right\rVert}
\newcommand{\transpose}{\intercal}

\newcommand{\conj}[1]{\bar{#1}}
\newcommand{\abs}[1]{\left|#1\right|}
\author{Jim Fowler}
\course{Math 6701}
\date{Week 6: Frobenius' theorem}

\begin{document}
\maketitle

We have previously seen that a vector field gives rise to integral
curves.  This is a fact that can also be interpreted as a statement
about coordinates: any vector field, in suitable local coordinates, is
$\partial_{x_1}$.

What if we have multiple vector fields?  Is it always the case, given
a couple vector fields, that we can find local coordinates
$(x_1,x_2,\ldots)$ so that these two vector fields are
$\partial_{x_1}$ and $\partial_{x_2}$?  The bracket serves as an
obstruction to doing so.  Our study of multiple vector fields leads to
generalizing integral curves to \textbf{integral manifolds}, where we
consider subbundles of the tangent bundle---here, called
\textbf{distributions}---and attempt to foliate our manifold by
submanifolds whose tangent bundles are exactly those distributions,
i.e., we seek conditions on which the distribution is
\textbf{completely integrable}.  The theorem behind this story is the
celebrated \textbf{Frobenius' theorem} and serves as our goal this
week.

From Morita's \textit{Geometry of Differential Forms}, read
\begin{itemize}
\item 2.3 Frobenius theorem
\item 2.4 A few facts
\end{itemize} Frobenius' theorem is a powerful generalization of the
uniqueness and existence theorem for ODEs, and makes good use of the
terminology we have painstakingly built in this course.  In
particular, differential forms provide a useful perspective, dual to a
``bare hands'' perspective involving the bracket of vector fields.
Having seen formulas which relate the bracket to exterior
differention, perhaps these relationships do not seem so surprising to
you now.

And yet, Frobenius' theorem very power and generality can make it seem
useless.  Where will I ever use a theorem of such generality?  The
problem set invites you to consider the special case of Lie subgroups,
where the involutive hypothesis of Frobenius' theorem is suddenly the
most natural thing in the world.  And digging deeper into Lie
subgroups will reinforce the difference between embeddings and
immersions, making connections with our prior learning.


\end{document}
